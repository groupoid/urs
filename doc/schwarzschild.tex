\documentclass{article}
\usepackage{amsmath, amssymb, amsthm}
\usepackage{parskip}
\usepackage{enumitem}

\theoremstyle{plain}
\newtheorem{theorem}{Theorem}
\newtheorem{lemma}{Lemma}
\newtheorem{proposition}{Proposition}

\newcommand{\supercoords}{z^{M}}
\newcommand{\boscoords}{x^{\mu}}
\newcommand{\fermcoords}{\theta^{a}}
\newcommand{\superline}{ds^{2}}
\newcommand{\schwarzschild}{g_{\mu\nu}}
\newcommand{\supervielbein}{E^{A}}
\newcommand{\superconnection}{\omega_{A}{}^{B}}
\newcommand{\superriemann}{R_{ABCD}}
\newcommand{\lie}[1]{\mathcal{L}_{#1}}
\newcommand{\dvol}{\mathrm{dVol}}


\begin{document}

\title{The Schwarzschild Metric in Supergeometry}
\author{Namdak Tonpa}
\date{May 9, 2025}

\maketitle

\begin{abstract}
This article explores the Schwarzschild metric within the framework of supergeometry, extending the classical solution of general relativity to a supersymmetric context. We define the metric on a $(4|2)$-dimensional supermanifold and derive its superline element using the supervielbein formalism. Key mathematical properties are established through theorems, including supersymmetric invariance and curvature properties. We employ algebraic tools such as Lie superalgebras, superdifferential forms, and BRST cohomology to study the geometry, and discuss integration techniques like the Berezin integral for fermionic coordinates.
\end{abstract}

\tableofcontents


\section{The Schwarzschild Metric in Supergeometry}

The Schwarzschild metric describes the spacetime geometry around a non-rotating, spherically symmetric mass in general relativity. Supergeometry, the mathematical foundation of supersymmetric theories like supergravity, extends spacetime to a supermanifold with bosonic and fermionic coordinates. This article formalizes the Schwarzschild metric in supergeometry, derives its properties, and applies algebraic tools to analyze its structure. We focus on theorems, derivations, integration, and BRST cohomology, providing a rigorous mathematical framework.

\subsection{Supergeometry and the Schwarzschild Metric}

% Defining the supermanifold
A supermanifold of dimension $(m|n)$ has $m$ bosonic coordinates $\boscoords$ and $n$ fermionic coordinates $\fermcoords$. We consider a $(4|2)$-dimensional supermanifold with coordinates $\supercoords = (t, r, \theta, \phi, \theta^{1}, \theta^{2})$, where $(t, r, \theta, \phi)$ are bosonic and $(\theta^{1}, \theta^{2})$ are Grassmann-odd.

% Recalling the Schwarzschild metric
The Schwarzschild metric in general relativity is:
\begin{equation}
\superline = -\left(1 - \frac{2GM}{r}\right) dt^{2} + \left(1 - \frac{2GM}{r}\right)^{-1} dr^{2} + r^{2} (d\theta^{2} + \sin^{2}\theta \, d\phi^{2}),
\label{eq:schwarzschild}
\end{equation}
where $G$ is the gravitational constant and $M$ is the mass.

% Extending to supergeometry
In supergeometry, the superline element is:
\begin{equation}
\superline = g_{MN}(\supercoords) dz^{M} dz^{N},
\end{equation}
where $g_{MN}$ is a superfield, and indices $M, N$ run over bosonic ($\mu, \nu$) and fermionic ($a, b$) coordinates. We assume a minimal extension where:
\begin{equation}
g_{\mu\nu}(x, \theta) = \schwarzschild(x) + \mathcal{O}(\theta), \quad g_{\mu a} = 0, \quad g_{ab} = \eta_{ab},
\end{equation}
yielding:
\begin{equation}
\superline = -\left(1 - \frac{2GM}{r}\right) dt^{2} + \left(1 - \frac{2GM}{r}\right)^{-1} dr^{2} + r^{2} (d\theta^{2} + \sin^{2}\theta \, d\phi^{2}) + \eta_{ab} d\theta^{a} d\theta^{b}.
\label{eq:superline}
\end{equation}

\subsection{Supervielbein Formalism}

% Introducing the supervielbein
The supervielbein $\supervielbein = dz^{M} E_{M}{}^{A}$ defines the supergeometry, where $A = (a, \alpha)$ includes bosonic ($a$) and fermionic ($\alpha$) indices. The superline element is:
\begin{equation}
\superline = \supervielbein \eta_{AB} E^{B},
\end{equation}
where $\eta_{AB} = \mathrm{diag}(-1, 1, 1, 1, 1, -1)$ is the flat superspace metric. For the Schwarzschild metric, we take:
\begin{equation}
E^{a} = e^{a}{}_{\mu}(x) dx^{\mu}, \quad E^{\alpha} = d\theta^{\alpha},
\end{equation}
where $e^{a}{}_{\mu}$ is the standard vielbein for \eqref{eq:schwarzschild}, and the fermionic sector is flat.

% Deriving the supervielbein components
The bosonic vielbein components are:
\begin{align}
e^{0} &= \sqrt{1 - \frac{2GM}{r}} \, dt, \quad e^{1} = \left(1 - \frac{2GM}{r}\right)^{-1/2} dr, \\
e^{2} &= r \, d\theta, \quad e^{3} = r \sin\theta \, d\phi.
\end{align}
The superline element \eqref{eq:superline} is recovered by computing $\supervielbein \eta_{AB} E^{B}$.

\subsection{Mathematical Properties}

% Establishing key theorems
We present theorems to characterize the Schwarzschild supergeometry.

\begin{theorem}[Supersymmetric Invariance]
The superline element \eqref{eq:superline} is invariant under the supersymmetry transformation:
\begin{equation}
\delta x^{\mu} = \epsilon^{a} \Gamma^{\mu}_{a}, \quad \delta \theta^{a} = \epsilon^{a},
\end{equation}
where $\epsilon^{a}$ are Grassmann-odd parameters, and $\Gamma^{\mu}_{a}$ are zero in the minimal extension.
\end{theorem}

\begin{proof}
Since $g_{\mu a} = 0$ and $g_{ab} = \eta_{ab}$ are constant, and $g_{\mu\nu}$ depends only on $x^{\mu}$, the transformation $\delta \theta^{a} = \epsilon^{a}$ leaves the fermionic sector invariant, and $\delta x^{\mu} = 0$ ensures the bosonic metric is unchanged.
\end{proof}

\begin{proposition}[Curvature of the Bosonic Sector]
The Riemann curvature tensor of the bosonic sector of \eqref{eq:superline} matches that of the standard Schwarzschild metric, with non-zero components proportional to $(GM)/r^{3}$.
\end{proposition}

\begin{proof}
The bosonic metric $g_{\mu\nu}$ is identical to \eqref{eq:schwarzschild}. Standard computations of the Riemann tensor $R_{\mu\nu\rho\sigma}$ yield the known Schwarzschild curvature, unaffected by the flat fermionic sector.
\end{proof}

\subsection{Integration in Supergeometry}

% Discussing the Berezin integral
Integration over the supermanifold involves the Berezin integral for fermionic coordinates. For a superfield $f(x, \theta)$, the integral is:
\begin{equation}
\int f(x, \theta) \, \dvol = \int dx^{4} \int d\theta^{1} d\theta^{2} \, f(x, \theta),
\end{equation}
where the Berezin integral is defined as:
\begin{equation}
\int d\theta^{1} d\theta^{2} \, (f_{0}(x) + f_{1}(x) \theta^{1} + f_{2}(x) \theta^{2} + f_{12}(x) \theta^{1} \theta^{2}) = f_{12}(x).
\end{equation}
For the Schwarzschild super metric, we compute the volume form:
\begin{equation}
\dvol = \sqrt{|\det g_{MN}|} \, dx^{0} \wedge \cdots \wedge dx^{3} \wedge d\theta^{1} \wedge d\theta^{2} = r^{2} \sin\theta \, dt \wedge dr \wedge d\theta \wedge d\phi \wedge d\theta^{1} \wedge d\theta^{2},
\end{equation}
since $\det g_{\mu\nu} = -r^{4} \sin^{2}\theta$ and $\det g_{ab} = 1$.

% Example application
\begin{proposition}
The integral of the superfield $f(x, \theta) = g_{tt}(r) + \theta^{1} \theta^{2}$ over a hypersurface $t = \text{const}$ vanishes unless fermionic terms are present.
\end{proposition}

\begin{proof}
The Berezin integral extracts the $\theta^{1} \theta^{2}$ coefficient, yielding $\int d\theta^{1} d\theta^{2} \, f = 1$. The bosonic integral $\int r^{2} \sin\theta \, dr \wedge d\theta \wedge d\phi$ is finite over a bounded region, confirming non-zero contributions only from fermionic terms.
\end{proof}

\subsection{BRST Cohomology}

% Introducing BRST cohomology
In supergravity, BRST cohomology quantizes gauge symmetries, including diffeomorphisms and supersymmetry transformations. The BRST operator $Q$ acts on fields, with $Q^{2} = 0$. For the Schwarzschild super metric, we consider the ghost fields $c^{\mu}$ (bosonic) and $c^{a}$ (fermionic) for diffeomorphisms and supersymmetry.

% Defining the BRST transformation
The BRST transformations are:
\begin{align}
Q x^{\mu} &= c^{\mu}, \quad Q \theta^{a} = c^{a}, \\
Q g_{MN} &= \lie{c} g_{MN} = c^{P} \partial_{P} g_{MN} + (\partial_{P} c^{Q}) g_{QN} + (\partial_{Q} c^{P}) g_{MP},
\end{align}
where $\lie{c}$ is the Lie derivative along the ghost vector $c = c^{M} \partial_{M}$.

\begin{theorem}[BRST Invariance]
The superline element \eqref{eq:superline} is BRST invariant up to a total derivative.
\end{theorem}

\begin{proof}
The variation $Q \superline = \lie{c} \superline$ is the Lie derivative of the metric. Since $\lie{c} g_{MN}$ transforms as a tensor, the superline element is invariant under BRST transformations, modulo boundary terms.
\end{proof}

% Cohomology classes
The BRST cohomology groups $H^{n}(Q)$ classify physical states. For the Schwarzschild supergeometry, non-trivial cohomology arises from supersymmetric Killing vectors, discussed below.

\subsection{Algebraic Tools}

% Lie superalgebras
The symmetry algebra of the supermanifold is a Lie superalgebra, combining Poincaré transformations with supersymmetry generators. The super-Poincaré algebra includes bosonic generators $P_{\mu}, M_{\mu\nu}$ and fermionic generators $Q_{\alpha}$:
\begin{equation}
[P_{\mu}, Q_{\alpha}] = 0, \quad \{ Q_{\alpha}, Q_{\beta} \} = \gamma^{\mu}_{\alpha\beta} P_{\mu}.
\end{equation}
The Schwarzschild super metric admits Killing supervectors, satisfying:
\begin{equation}
\lie{\xi} g_{MN} = 0,
\end{equation}
where $\xi = \xi^{\mu} \partial_{\mu} + \xi^{a} \partial_{a}$.

% Superdifferential forms
Superdifferential forms generalize differential forms to include fermionic directions. The volume form is a $(4|2)$-form:
\begin{equation}
\omega = r^{2} \sin\theta \, dt \wedge dr \wedge d\theta \wedge d\phi \wedge d\theta^{1} \wedge d\theta^{2}.
\end{equation}
The exterior derivative $d$ satisfies $d^{2} = 0$, and cohomology classes of the supermanifold can be computed using de Rham cohomology extended to superforms.

% Killing supervectors
\begin{proposition}
The bosonic sector of \eqref{eq:superline} admits four Killing vectors (time translation and SO(3) rotations), and the fermionic sector admits trivial Killing supervectors in the minimal extension.
\end{proposition}

\begin{proof}
The Schwarzschild metric has Killing vectors $\partial_{t}$,
$\partial_{\phi}$, $\sin\phi \partial_{\theta} + \cot\theta \cos\phi \partial_{\phi}$,
$\cos\phi \partial_{\theta} - \cot\theta \sin\phi \partial_{\phi}$.
The flat fermionic metric $\eta_{ab}$ admits no non-trivial Killing
supervectors since $\theta^{a}$ transformations are constant.
\end{proof}

\subsection{Conclusion}

% Summarizing the contribution
We have extended the Schwarzschild metric to supergeometry, deriving its superline element and exploring its mathematical properties through theorems, integration, BRST cohomology, and algebraic tools. The supervielbein formalism, Berezin integrals, and Lie superalgebras provide a rich framework for studying supersymmetric gravity. Future work could explore non-trivial superfield expansions or quantum corrections via BRST quantization.

% Including a bibliography
\begin{thebibliography}{9}
\bibitem{witten}
  Witten, E., ``Supergravity and Supermanifolds,'' \emph{Advances in Theoretical Physics}, 1983.
\bibitem{dewitt}
  DeWitt, B., ``Supermanifolds,'' \emph{Cambridge University Press}, 1992.
\bibitem{henneaux}
  Henneaux, M., Teitelboim, C., ``Quantization of Gauge Systems,'' \emph{Princeton University Press}, 1992.
\end{thebibliography}

\end{document}
